\documentclass[12pt]{beamer}
\usetheme{Warsaw}
\usepackage[utf8]{inputenc} % à utiliser pour overleaf
\usepackage[french]{babel}
\usepackage[T1]{fontenc}
%\usepackage{amsmath}
%\usepackage{amsfonts}
%\usepackage{amssymb}
%\usepackage{graphicx} %pour les images
\author[The Picross team]{Alexandre Kirchmeyer - Dorian Biichlé - Benjamin Loison - Léo Mangel - Florent Pollet}
\title[Picross SAT Solver]{SAT Solver :\\Picross}
\institute[Mathinfoly]{Mathinfoly}
%\date{\now} 
%\setbeamercovered{transparent} 
%\setbeamertemplate{navigation symbols}{} 
%\logo{} 
%\institute{} 
%\date{} 
%\subject{} 

\usepackage{style}
\usepackage{listings} % ou minted, à voir
\usefonttheme[onlymath]{serif}
\usepackage{setspace}

\usepackage{subfigure}

\makeatletter
\setbeamertemplate{theorem begin}
{%
\begin{\inserttheoremblockenv}
{%
\ifx\inserttheoremaddition\@empty\else\ \inserttheoremaddition\fi%
}%
}
\setbeamertemplate{theorem end}{\end{\inserttheoremblockenv}}
\makeatother


\setbeamertemplate{caption}{\raggedright\insertcaption\par}

\begin{document}


% Rien d'autre à faire qu'afficher le titre
\begin{frame}
\titlepage 
\locfigb{img/license.png}{}{i0}{0.3}
\end{frame}




\section[]{Introduction} %justification des choix, fil directeur

\begin{frame}
\frametitle{Introduction}
\framesubtitle{Laissez-vous guider !}


\begin{columns}
\column{0.5\textwidth}
\locfigb{img/toad.PNG}{Un joli puzzle}{i8}{0.9}

\column{0.5\textwidth}
\begin{itemize}
    \item C'est amusant !
    \item C'est parti !
\end{itemize}
\end{columns}

\end{frame}


% La table des matières utilise ce que vous donnez aux commandes \section et 
% \subsection tout au long de la présentation.
\begin{frame}
\frametitle{Plan} 
\tableofcontents 
\end{frame}


\section[]{Le jeu}
\begin{frame}
\frametitle{Le jeu}
\begin{itemize}
    \item Un puzzle = un hanjie = un nonogram
    \item 2 couleurs
    \item Contraintes sur les lignes et les colonnes : blocs de la longueur indiquée, dans l'ordre, séparé par au moins une case de l'autre couleur
    \item Une ou plusieurs solutions (selon les contraintes)
\end{itemize}

\locfigb{img/Hanjie.JPG}{Avec un papier et du crayon...}{i0}{0.4}


\end{frame}



\section[]{Modélisation}
\begin{frame}
\frametitle{Modélisation - Premier modèle}

\begin{itemize}
    \item Modèle naïf : une variable par case, génération de toutes les possibilités. 
    \item Utilisation de sympy pour le passage en CNF.
    \locfigb{img/config_to_expr.jpg}{Ecriture des conditions sous forme logique.}{i0}{0.7}
\end{itemize}

\end{frame}

\begin{frame}
\frametitle{Modélisation - Premier modèle}
    \locfigb{img/modele_naif.JPG}{Bruteforce des familles.}{i0}{0.8}
\end{frame}


\begin{frame}
\frametitle{Modélisation - Second modèle}
    \locfigb{img/second_modele.jpg}{Utilisation de variables secondaires}{i0}{0.8}

\end{frame}
\section[]{Complexité}
\begin{frame}{Complexité}
    \locfigb{img/complexite.jpg}{}{i0}{0.8}
\end{frame}


\section[]{Performances}
\begin{frame}
\frametitle{Performances}
%screen à mettre
Utilisation de boardgen
\begin{table}[]
\begin{tabular}{lllll}
i & Size & Generation time & Glucose time & Density \\
1 & 20     & 1.115s                 & 0.965s           & 0.5     \\
2 & 20     & 1.156s                 & 2.919s           & 0.475   \\
3 & 20     & 10.416s                & 2.33             & 0.45    \\
4 & 20     & 7.239s                 & 2.7s             & 0.425   \\
5 & 20     & 26.811s                & 10.103s          & 0.4     \\
6 & 30     & 15.58s                 & 73.094s          & 0.5     \\
7 & 30     & 10.81s                 & 17.215s          & 0.5     \\
8 & 33     & 641.249s              & 1861.86s        & 0.5    
\end{tabular}
\end{table}


\end{frame}

\section[]{Démonstration du programme}
\begin{frame}
\frametitle{Démonstration du programme}%0 ligne de code %explication jeu, modélisation, en tête dimacs, performance, imp?? , entrée, sortie
\begin{itemize}
    \item Fonctionnement avec pypy & utilisation de boardgen.py
    \item Cf à la fin pour le voir en direct !
\end{itemize}

%mettre un fichier d'entrée format

\begin{columns}
\column{0.5\textwidth}
\locfigb{img/math1.png}{Quelque chose de formidable}{i0}{0.6}
\column{0.5\textwidth}
\locfigb{img/mathinfoly.JPG}{Solution non unique (n et y)}{i0}{0.6}
\end{columns}
\end{frame}

\begin{frame}
\frametitle{Démonstration du programme}%0 ligne de code %explication jeu, modélisation, en tête dimacs, performance, imp?? , entrée, sortie

\begin{itemize}
    \item Détection des puzzles sans solutions
\end{itemize}


\begin{columns}
\column{0.5\textwidth}
\locfigb{img/bug.png}{Un exemple}{i0}{0.8}
\column{0.5\textwidth}
\locfigb{img/code.png}{Le code}{i0}{0.4}
\end{columns}

\end{frame}


%\begin{frame}
%\frametitle{Performances et étude de la complexité}
%\locfigb{}{Un benchmark}{i0}{0.4}
%\end{frame}


\section[]{Conclusion}
\begin{frame}
\begin{itemize}
    \item Recherche très intéressante
    \item Des progrès sont encore possibles :
\locfigb{img/contest.PNG}{Nonograms 25x25}{i9}{0.7}
    \item Merci pour votre attention !
\end{itemize}
\frametitle{Conclusion}


\end{frame}


\end{document}
